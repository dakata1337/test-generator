\subsection{Добавяне на библиотеки}

\subsubsection{Конзолни аргументи}
Аргументите на конзолата са параметри, предавани на програма преди изпълнение
на поргамата в командния ред. В Rust аргументите могат да бъдат достъпени чрез
функцията std::env::args(), която връща итератор над аргументите като списък от
низове.

За да вземем подходящата информация за приложението може да напишм наш собствен
анализатор или да изплозваме една от многото различни библиотеки за работа с
конзолни аргументи. Една от най-използваните библиотеки е clap (Console Line
Argument Parser).

Clap ни предоставя с clap::Parser макрото, което при компилирането на програмата
анализира структурата от данни и автоматично търси командните аргументи при екзекуция.
Също тъка проверява кои аргументи са маркирани като задължителни или такива със
стойност по подразбиране.
% TODO: add image

При въвеждане на грешни аргументи или при липсата на задължителните такива, Clap
показва автоматично генерираното помощно съобщение на потребителя.

% TODO: add image

\subsubsection{Serde}
Serde е framework за ефективно сериализиране и десериализиране на структури от данни в Rust.

Екосистемата на Serde се състои от структури от данни, които знаят как да
сериализират и десериализират себе си заедно с формати на данни, които знаят
как да сериализират и десериализират други неща. Serde предоставя слоя, чрез
който тези две групи взаимодействат помежду си, позволявайки всяка поддържана
структура от данни да бъде сериализирана и десериализирана с помощта на всеки
поддържан формат на данни.

Докато много други езици разчитат на runtime среда (като Dotnet) за
сериализиране на данни, Serde вместо това е изградена върху много добрата
интерфес система на Rust.

Структура от данни, която знае как да сериализира и десериализира сама себе си,
е тази, която използва интерфейсите на Serde за сериализиране и десериализиране
(или използва атрибута derive на Serde за автоматично генериране на интерфеси
по време на компилация). По този начин се избягват забавянето от употребата на
runtime среда.

Всъщност в много ситуации взаимодействието между структурата на данните и
формата на данните може да бъде напълно оптимизирано от Rust компилатора,
оставяйки сериализацията на Serde да се изпълни със същата скорост като
ръно написан сериализатор в езици от по-ниско ниво като C.

\subsubsection{Toml}
Toml (Tom's Obvious Minimal Language) е файлов формат за съхранение на
софтуерни конфигурации. Този формат ще бъде използван за съхраняване на
настройките, въпросите и друга информация за тестовете. За да добавим подръжка за
този формат трябва в Rust, трябва да инсталираме Toml библиотека която използва
Serde са преобразуването на файл в обект и обратно.

% TODO: explain how toml is integrated with serde
% TODO: exmaple?

\subsubsection{genPDF}
genPDF е библиотека за генериране на PDF документи на високо ниво, изградена от
две по-малки библиотеки от по-ниско ниво. Тя се грижи за оформлението на
страницата и подравняването на текста и изобразява дървовидна структура на
документа в PDF документ.
