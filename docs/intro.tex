\section{Увод}

В настоящата дисертация ще бъде представен проект, който има за цел да улесни
генерирането на изпитни билети за учебни заведения. Програмата е изградена на
езика Rust и предлага лесен и ефективен начин за създаване на изпитни билети.

За да може един софтурен продукт да бъде завършен на време, трябва да задачите
и целите на продукта да бъдат разделени на по-малки под задачи.
% ChatGPT:
% Има много различни методологии за разработване на софтуер. Waterfall Model:
% Waterfall моделът е традиционна методология за разработка на софтуер, където
% процесът протича последователно, с една фаза, която се завършва преди да
% започне следващата. Състои се от фази на събиране на изисквания, дизайн,
% изпълнение, тестване, развой и поддръжка.
%
% Spiral Model: Spiral моделът е методология за разработка на софтуер,
% която се управлява от риска и комбинира елементи от Waterfall и Agile
% методологиите. Включва няколко итерации на планиране, проектиране,
% изграждане и тестване, като всяка итерация се увеличава в сложност.
% 
% Lean Software Development: Lean Software Development методологията има за
% цел да намали загубите и да подобри ефективността в процеса на разработка
% на софтуер. Тя подчертава важността на непрекъснато подобряване,
% елиминиране на излишък, оптимизиране на цялата система и предоставяне на
% стойност на клиентите възможно най-рано.
% 
% Extreme Programming (XP): Extreme Programming е вид Agile методология,
% която се фокусира върху удовлетворението на клиентите, работата в екип и
% гъвкавостта. Тя включва практики като двойно програмиране, непрекъсната
% интеграция и разработка, базирана на тестове.
% 
% Kanban: Kanban е визуална рамка за управление на проекти, която помага на
% екипите да управляват и подобряват работния си процес. Тя подчертава
% важността на ограничаването на работата в ход, непрекъснато доставяне и
% визуализиране на потока на работа.
% 
% DevOps: DevOps е методология, която се фокусира върху сътрудничеството
% между екипите за разработка и операции с цел подобряване на скоростта и
% качеството на доставката на софтуер. Тя включва непрекъсната интеграция,
% непрекъсната доставка и непрекъснато наблюдение, за да се гарантира, че
% софтуерът с

% TODO: agile, scrum
% TODO: жинен цикъл

% ChatGPT:
% В дисертацията ще бъдат представени подробности относно организацията на
% проекта, както и какъв методологичен подход е използван по време на
% разработката му. За да бъде постигната висока ефективност на проекта, беше
% използван Agile подход и SCRUM модел, които допринасят за по-добро управление
% на екипа и по-ефективно използване на ресурсите.
%
% Важен аспект при разработката на софтуер е неговият животен цикъл и поддръжката
% му. В тази дисертация ще бъдат разгледани детайли относно поддръжката и
% разширяемостта на програмата, за да може да се гарантира дългосрочната ѝ
% работоспособност и удовлетворение на нуждите на потребителите.

