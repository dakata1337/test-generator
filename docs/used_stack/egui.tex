\subsection{egui}
egui е проста, бърза и много преносима библиотека за графични потребителски
интерфейси (GUI). Тя работи на много платформи включително: уеб браузъри, като
обикновено приложение или в някои game engine-а. Написана е на Rust и има много
лесен и интуитивен API за разработване.

Главните цели на проекта са:
\begin{itemize}
    \item Най-лесната за използване GUI библиотека;
    \item Отзивчив: цели поне 60 FPS при компилация с Debug опциите;
    \item Преносим: кодът да работи в браузър и като собствено приложение;
    \item Лесен за интегриране във всяка среда;
    \item Модулен: можете да използвате малки части от egui и да ги комбинирате
    по нови начини \item Минимален брой завивисимости (библиотеки)..
\end{itemize}

\subsubsection{Минимален пример}
egui библиотеката ни дава достъп до \textit{eframe::App} интерфейса. Този интерфес съдържа една
функция \textit{update}. Тя се извиква всеки пътр когато потребителският интерфейс се е
променил или се получи някакво събите (Event) от мишката или клавиатурата. 
[Фигура \ref{fig:egui-example-1}]

\begin{figure}[!htb]
  \includegraphics[width=\textwidth,keepaspectratio=true]{egui-example-1}
  \centering
  \caption{Имплементация на egui интерфейса}
  \label{fig:egui-example-1}
\end{figure}

Библиотеката е достатъчно умна сама да прецени дали се нуждае от повторно
изобразяване, или не.

egui е GUI библиотека от незабавен режим (Immediate Mode). Това означава, че
начина, по който искаме да изглежда графичния интерфейс, се описва, извиквайки
методи. По този начин се упростява разработката на графичния интерфейс. За да покажем един прост
бутон се нуждаем от един if оператор.
