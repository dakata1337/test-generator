\subsection{genpdf}
genpdf е библиотека, която абстрахира създаването на PDF файлове. Тя се грижи за
оформлението на страницата, подравняването на текста и изобразяването на
структурата на документа в PDF файл. Библиотекатар както и всичките ѝ зависимости
са написани на Rust и следват добрите практики на езика
\cite{genpdf_repository}.

genpdf използва елементи, за да опише оформлението на документа. Всеки елемент
имплементира \textit{genpdf::Element} интерфейса. Интерфейсът съдаржа функция \textit{render}, която бива
извикана всеки път когато елементът трябва да бъде показан в PDF файла
\cite{genpdf_element_trait}.

Използвайки \textit{genpdf::Element} интерфейса, разработчиците на genpdf са ни предоставили
най-често използваните елементи:
\begin{itemize}
    \item Контейнери:
        \subitem LinearLayout: подрежда елементите си последователно;
        \subitem TableLayout: подрежда елементите си в колони и редове;
        \subitem OrderedList/UnorderedList: подредете елементите им последователно с bullet-и.
    \item Текст:
        \subitem Text: един ред текст;
        \subitem Paragraph: подравнен параграф.
    \item Обвивки:
        \subitem FramedElement: елемент с рамка;
        \subitem PaddedElement: добавя разтояние между елементите;
        \subitem StyledElement: задава стил по подразбиране за обвития елемент и неговите деца (елементите, които му принадлежат).
    \item Други:
        \subitem Image: снимка с описание;
        \subitem Break: добавя прекъсвания на редове като разделител;
        \subitem PageBreak: добавя принудително прекъсване на страницата.
\end{itemize}
