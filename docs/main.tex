\section{Използвани езици и технологии}

\subsection{Rust}
Rust е програмен език от високо ниво създаден през 2006 година от Graydon
Hoare, който по това време работи за Mozilla. През 2009 година разработката на
езика бива спонсорирана от Mozilla, a през 2010 езика е обявен публично.
\cite{Rust_Origins_Wikipedia}

\subsubsection{Отличаващи се особености на езика}
% Rust е порграмен език от високо ниво, но това не му пречи да е почти толкова
% бърз колкото езици от по-ниско ниво като C/C++.

Езици като C\#, Python и JavaScript използват система за освобождаване на паметта
наречена Garbage Collector (GC). За да може да се освободят неизползваните
променливи, изпълнението на програмата трябва да бъде спряно на пауза и да се
провери дали има заделени региони от паметта, към които вече не се използват
или са маркиране за освобождаване от програмиста \cite{Garbage_Collection_Wikipedia}

Rust използва система наречена borrow checker, която проверява, по време на
компилация, дали програмата следва следните принципи:

\begin{itemize}

\item Ресурсите (отделената памет за стойноста) могат да имат само един
собственик и това е самата промелива. Когато променлива вече не може да бъде
достъпена тя бива освободена.

\item Когато една променлива бъде подадена към някоя функция, собственик на
ресурсите става променливата във функцията. Ако се пробваме подадем отново
променливата, компилатора ще ни каже, че променливата е преместена (Use of
moved value).
% TODO: put image of `Use of moved value` error

\end{itemize}

\subsubsection{Enum}
Enum е един от основните типове в Rust. Всеки вариянт на enum-а може да има съдържа информация от различен вид \cite{Rust_Enums}. Така са имплементирани някои от най-важните типове: Option<T> и Result<T, E> [Фигура \ref{fig:rust-enum-example}].

\subsubsection{Option типа}
В повечето езици съществува идеята за NULL пойнтери. Когато един pointer е Null
това означа, че той сочи към нищо. Идеята за Null на теория е много добра, но
на практика създава повече проблеми. Ако се пробваме да достъпим pointer който
е Null, програмата ще крашне или в някои езици като C\# ще хвърли
NullReferenceException.

Разработчиците на Rust са намерили много добър заместител на Null и това е
Option enum-а, който има два варинта. Това са Some(T) когато имаме някаква
стойнос и None когато нямаме нищо.

\subsubsection{Result типа}
Когато програмираме на C\# много често ни се случва да хвърляме Exception-и и
съответно да ги хващаме с try/catch блока. Еxception-ите се ползват когато в
една функция възникне грешка.

Във Фигура \ref{fig:csharp-exceptions-1-code} е даден код който на пръв поглед
изглежда добре, но има скрити бъгове. Какво ще стане ако потребилтеля въведе
дума вместо число? Ще получим Exception който ни казва: "Input string was not
in a correct format" [Фигура \ref{fig:csharp-exceptions-1-output}].

Проблема е че ние като програмисти не знаем, че int.Parse може да хвърли
Exception без да се консултираме с документацията \cite{CSharp_Int_Parse}.
Същият код написан на Rust би изглеждал по следния начин [Фигура \ref{fig:rust-exceptions-1-code}].

Разликата между C\# и Rust е че Rust кода ни кара ни показва типовете при успех
и грешка. Функцията връща променлива от тип Result<T, E> където T е
променливата от тип i32 (int) ако всичко се и изпълнило без проблем, а E е от
тип ParseIntError.

За да използваме резултата от функцията, какъвто и да е той, можем да
използваме match. С match можем да проверим дали резултата е Ok или Err.

% \subsection{PDF}
% PDF е файлов формат създаден от Adobe през 1992г. предназначен да бъде преносим
% документ, независим от хардуера и софтуера. Всеки един PDF файл съдържа пълно
% описание на начина по който трябва да изглежда документа \cite{PDF_Wikipedia}.
%
% \subsubsection{Технически детайли}
% PDF най-често е комбинация от векторни графики и текст.
%
% В по-нови ревизии на PDF станзарта, документите могат да имат линкове които можем да отворим и подръжка за plugin-и.

\subsection{egui}
egui (pronounced "e-gooey") is a simple, fast, and highly portable immediate
mode GUI library for Rust. egui runs on the web, natively, and in your favorite
game engine (or will soon).

egui е проста, бърза и много преносима библиотека за графични потребителски
интерфейси. Egui работи на много платформи включително: уеб браузъри, като
обикновено приложение и в някои game engine-а. Написана е на Rust и има много
лесен и интуативен API за разработване.

Главните цели на проекта са:
\begin{itemize}
    \item Най-лесната за използване GUI библиотека
    \item Отзивчив: целеви 60 Hz при компилация с Debug опциите
    \item Преносим: същият код работи в браузър и като собствено приложение
    Easy to integrate into any environment
    A simple 2D graphics API for custom painting (epaint).
    No callbacks
    Pure immediate mode
    Extensible: easy to write your own widgets for egui
    Modular: You should be able to use small parts of egui and combine them in new ways
    Safe: there is no unsafe code in egui
    Minimal dependencies: ab_glyph ahash nohash-hasher parking_lot
\end{itemize}

